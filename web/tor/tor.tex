\documentclass[12pt]{report}

\usepackage[russian]{babel}
\usepackage{geometry}
\usepackage{setspace}
\geometry{a4paper, left=2cm, right=1cm, top=2cm, bottom=2cm}

\usepackage[pdfauthor={Ларионов Александр Сергеевич},
            pdftitle={Проект "Финансовый Web-помощник"},
            pdfsubject={Web-разработка},
            pdfkeywords={лабораторная работа, программная инженерия, web, проект},
            pdfproducer={LaTeX with hyperref},
            pdfcreator={pdflatex}]{hyperref}

\onehalfspacing

\begin{document}

\thispagestyle{empty}
\begin{center}
\renewcommand{\baselinestretch}{1}
\large
{\sc Министерство образования и науки Российской Федерации\\
ФГБОУ <<Петрозаводский государственный университет>>\\
Институт математики и информационных технологий\\
Кафедра информатики и математического обеспечения}

\end{center}

\begin{center}
    09.03.04 -- Программная инженерия
\end{center}

\vfill
\vspace{5cm}
\begin{center}
{\normalsize Техническое задание проекта} \\

\medskip

{\Large \sc {<<Финансовый Web-помощник>>}} \\
\end{center}
\bigskip
\vfill
\vspace{9cm}
\begin{flushright}
\parbox{6cm}{%
\normalsize
Выполнил:\\
студент 2 курса группы 22207\\
А.~С.~Ларионов\\
\par}
\end{flushright}

\vspace{2cm}

\vfill

\begin{center}
Петрозаводск 2025
\end{center}

\newpage

\section*{Техническое задание проекта "Финансовый Web-помощник"}

\subsection*{1. Общие сведения о проекте}

\textbf{Название проекта:} Финансовый Web-помощник

\textbf{Цель проекта:} Создание веб-сайта для учета личных финансов с элементами аналитики и геймификации.

\subsection*{2. Функциональные требования}

\subsubsection*{2.1. Общая структура сайта}
\begin{itemize}
    \item Шапка сайта: логотип, кнопки входа/регистрации
    \item Левая панель: навигация между разделами. Для гостей сайта ссылки "Тест" и "Аналитика и статистика" скрыты.
    \item Футер: контактная информация, лицензия
\end{itemize}

\subsubsection*{2.2. Основные страницы и функционал}

\textbf{2.2.1. Главная страница}
\begin{itemize}
    \item Таблица расходов с колонками: дата, название, категория, сумма
    \item Кнопка управления категориями (добавление/удаление через всплывающее окно)
    \item Кнопка управления записями расходов (добавление/удаление через всплывающее окно)
    \item Сортировка по колонкам (категории)
    \item Правая панель: краткая аналитика и статистика, панель есть ссылка на полную страницу с аналитикой
    \item Для неавторизованных пользователей - ограниченный доступ: таблица и панель аналитики "закрыта" предложением авторизации
\end{itemize}

\textbf{2.2.2. Мини-игра "Кешбекни!"}
\begin{itemize}
    \item Анимированная игра на Canvas/JavaScript
    \item Механика: ловля падающих чеков
    \item Система рекордов
    \item Гостевой режим для неавторизованных пользователей без системы рекордов
\end{itemize}

\textbf{2.2.3. Страница тестирования}
\begin{itemize}
    \item Динамическая генерация тестов
    \item Различные типы вопросов: одиночный выбор, множественный выбор, выпадающий список, числовой ввод, текстовый ввод, соответствие
    \item Требуется авторизация для доступа
\end{itemize}

\textbf{2.2.4. Аналитика и статистика}
\begin{itemize}
    \item Графики расходов по дням
    \item Диаграммы по категориям
    \item Максимальные/минимальные расходы
    \item Настройка отображаемых данных
    \item Требуется авторизация для доступа
\end{itemize}

\textbf{2.2.5. Дополнительные страницы}
\begin{itemize}
    \item Страница "О проекте" с информацией и формой обратной связи
    \item Страница регистрации/авторизации
    \item Страница профиля пользователя
\end{itemize}

\subsection*{3. Технические требования}

\subsubsection*{3.1. Frontend}
\begin{itemize}
    \item HTML5, CSS3, JavaScript
    \item Canvas для реализации игры
    \item Адаптивный дизайн
\end{itemize}

\subsection*{4. Система доступа}

\begin{itemize}
    \item Авторизованные пользователи: полный доступ ко всем функциям
    \item Неавторизованные пользователи: ограниченный доступ к основной функциональности, гостевой режим игры
\end{itemize}

\end{document}