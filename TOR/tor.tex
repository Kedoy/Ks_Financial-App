\documentclass[12pt, a4paper]{article}
\usepackage[utf8]{inputenc}
\usepackage[T2A]{fontenc}
\usepackage[russian]{babel}
\usepackage{amsmath}
\usepackage{graphicx}
\usepackage{listings}
\usepackage{lipsum} % Пакет для создания "рыбного текста" - заменить на реальный текст
\graphicspath{ {./images/} } % Путь к папке с изображениями

\title{Отчет о разработке мобильного приложения \\ «Система учета расходов по SMS»}
\author{
    Исполнитель: <ФИО Исполнителя> \\
    Роль: <Ваша Роль> \\
    Дата: <Дата составления отчета>
}
\date{}

\begin{document}
\maketitle
\tableofcontents
\newpage

\section{Введение}
\label{sec:introduction}
\subsection{Цель и Задачи Проекта}
Целью данного проекта является разработка мобильного приложения для платформы Android, которое позволит пользователям автоматически отслеживать и анализировать свои финансовые расходы на основе SMS-сообщений от банков.
\lipsum[1] % Заменить на описание проблемы и ее решения

\subsection{Область Применения}
<PLACEHOLDER: Описание целевой аудитории и сферы применения.>
\lipsum[2]

\newpage
\section{Архитектура Приложения}
\label{sec:architecture}
\subsection{Общая Структура}
Проект использует клиент-серверную архитектуру, где мобильное приложение (клиент) взаимодействует с удаленным сервером (бэкенд) посредством RESTful API.

\begin{itemize}
    \item \textbf{Клиент (Android App):} Отвечает за интерфейс, считывание SMS, парсинг данных и отправку их на сервер.
    \item \textbf{Сервер (Backend):} Отвечает за хранение данных (БД), систему аутентификации, обработку запросов от клиента и предоставление аналитической информации.
\end{itemize}

\subsection{Диаграмма Архитектуры}


\begin{figure}[h]
    \centering
    \includegraphics[width=0.8\textwidth]{<PLACEHOLDER: file_name_architecture_diagram.png>} % Заменить на имя файла
    \caption{Схема архитектуры проекта}
    \label{fig:architecture}
\end{figure}

\newpage
\section{Компоненты Клиентской Части (Android)}
\label{sec:client_components}

\subsection{Классы и Модели Данных}
Описание ключевых классов, используемых в приложении для представления данных и логики.

\begin{itemize}
    \item \textbf{UserModel:} Модель данных для хранения информации о пользователе (ID, токен, настройки).
    \item \textbf{TransactionModel:} Модель данных для представления финансовой транзакции (Сумма, Категория, Дата, Описание).
    \item \textbf{SmsParserUtil:} Статический класс/утилита для извлечения информации из текстового тела SMS.
    \item \textbf{...} <PLACEHOLDER: Добавить другие ключевые классы>
\end{itemize}

\subsection{Сервисы и Логика}
Описание ключевых компонентов, работающих в фоновом режиме или реализующих основную логику.

\begin{itemize}
    \item \textbf{SmsReceiverService:} Сервис для прослушивания входящих SMS-сообщений.
    \item \textbf{AuthService:} Сервис для взаимодействия с API аутентификации (логин, регистрация).
    \item \textbf{DataSyncService:} Сервис для отправки спарсенных транзакций на бэкенд и получения истории расходов.
    \item \textbf{...} <PLACEHOLDER: Добавить другие ключевые сервисы>
\end{itemize}

\newpage
\section{Реализация Основной Функциональности}
\label{sec:implementation}

\subsection{Система Аутентификации}
Процесс регистрации и входа пользователя, использование токенов для защищенного доступа к API.
<PLACEHOLDER: Описание механизма аутентификации (например, OAuth2, JWT).>

\subsection{Парсинг SMS и Сохранение Данных}
Детализация процесса получения SMS, его фильтрации (по номеру телефона), парсинга суммы и других деталей, а также последующей отправки данных на сервер.

\subsubsection{Алгоритм Парсинга}
<PLACEHOLDER: Описание регулярных выражений или логики для извлечения суммы, даты, описания из SMS.>

\subsection{API Бэкенда}
Краткое описание ключевых конечных точек (endpoints) API.

\begin{itemize}
    \item \texttt{POST /api/auth/login}
    \item \texttt{POST /api/transactions/upload}
    \item \texttt{GET /api/transactions/history}
    \item \texttt{GET /api/analytics/summary}
\end{itemize}
<PLACEHOLDER: Детализация параметров и ответов.>

\newpage
\section{Скриншоты Приложения}
\label{sec:screenshots}
Визуальное представление разработанного приложения.

\subsection{Экран Авторизации/Регистрации}
\begin{figure}[h]
    \centering
    \includegraphics[width=0.45\textwidth]{<PLACEHOLDER: screenshot_login.png>}
    \caption{Экран входа}
    \label{fig:login}
\end{figure}

\subsection{Главный Экран с Расходами}
\begin{figure}[h]
    \centering
    \includegraphics[width=0.45\textwidth]{<PLACEHOLDER: screenshot_main.png>}
    \caption{Список транзакций}
    \label{fig:main}
\end{figure}

\subsection{Экран Аналитики}
\begin{figure}[h]
    \centering
    \includegraphics[width=0.45\textwidth]{<PLACEHOLDER: screenshot_analytics.png>}
    \caption{Визуализация расходов}
    \label{fig:analytics}
\end{figure}

\end{document}